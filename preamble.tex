%
% LaTeX2e Preamble / Ross Duncan 2011
%
%-------------------------------------------------------
% TODO 27/1/2011
% 1. Something for tikz
% 2. Font alternatives?
% 3. Which journals clash with which bits?
% 4. No support for logic -- proof trees linear logic symbols etc
% 5. Use switches to turn on/off some parts
%-------------------------------------------------------

% typographical improvements
\usepackage[T1]{fontenc}
\usepackage{lmodern}
\usepackage{fixltx2e}
\usepackage{microtype} 
%\usepackage[stretch=10]{microtype} % more conservative
%\usepackage[expansion=false]{microtype} % still more conservative
\usepackage{xspace}

% make figs a bit nicer
\newcommand{\figureline}{\rule{\textwidth}{0.5pt}}
\newcommand{\figureend}{\rule{\textwidth}{0.5pt}}

% do The Right Thing
\makeatletter
\newcommand\etc{etc\@ifnextchar.{}{.\@}\xspace}
\newcommand\ie{i.e.\@\xspace}  % these two may be broken
\newcommand\eg{e.g.\@\xspace}
\makeatother

% arg is height; optional arg is baseline shift
\newcommand{\vstrut}[2][0pt]{\rule[#1]{0pt}{#2}}

%%%%%%%%%%%%%%%%%%%%%%%%%%%%%%%%%%%%%%%%%%%%%%%%%%%%%%%
% Include graphics in a nice way
%%%%%%%%%%%%%%%%%%%%%%%%%%%%%%%%%%%%%%%%%%%%%%%%%%%%%%%
\usepackage{graphicx}

%%%% Use these commands to insert graphics into 
%%%% the text and have them line up properly.

%%% - inlinegraphic requires absolute size to be given
%%% - ninlinegraphic uses optional scaling factor

%%%% first param is the desired height of the graphic
%%%% if it is less the 1ex, it might not look good
\newcommand{\inlinegraphic}[2]{
  %% todo -- make this thing calculate the height 
  %% itself based on a global scaling factor
  \dimendef\grafheight=255\dimendef\grafvshift=254
  \grafheight=#1
  \grafvshift=-0.5\grafheight
  \advance\grafvshift by 0.5ex
  \raisebox{\grafvshift}{\includegraphics[height=\grafheight]{images/#2}\xspace}
}


%%% optional first arg is scaling factor
\newcommand{\ninlinegraphic}[2][1.0]{
  \dimendef\grafheight=255\dimendef\grafvshift=254
  \setbox0 = \hbox{\scalebox{#1}{\includegraphics{images/#2}}}
  \grafheight=\the\ht0
  \grafvshift=-0.5\grafheight
  \advance\grafvshift by 0.5ex
  \raisebox{\grafvshift}{\includegraphics[height=\grafheight]{images/#2}\xspace}
}

\newcommand{\inline}[1]{
  \raisebox{0.5ex}{\;#1\;}
}


%%%%%%%%%%%%%%%%%%%%%%%%%%%%%%%%%%%%%%%%%%%%%%%%%%%%%%%
%% Mathematical packages and theorems
%%%%%%%%%%%%%%%%%%%%%%%%%%%%%%%%%%%%%%%%%%%%%%%%%%%%%%%

\usepackage{latexsym}
\usepackage{amssymb}
\usepackage{amsmath} % can collide with some journals
\usepackage{stmaryrd}
\usepackage{mathtools}
%%% not needed if amsmath loaded
% \newcommand{\text}[1]{\ensuremath{\mbox{#1}}}


%%% Defining theorems and friends usually makes problems for journal
%%% styles
\usepackage{amsthm}
\newtheorem{theorem}{Theorem}[section]
\newtheorem{proposition}[theorem]{Proposition}
\newtheorem{lemma}[theorem]{Lemma}
\newtheorem{corollary}[theorem]{Corollary}
\theoremstyle{definition}\newtheorem{example}[theorem]{Example}
\theoremstyle{definition}\newtheorem{examples}[theorem]{Examples}
\theoremstyle{definition}\newtheorem{definition}[theorem]{Definition}
\theoremstyle{definition}\newtheorem{definitions}[theorem]{Definitions}
\theoremstyle{definition}\newtheorem{remark}[theorem]{Remark}
\theoremstyle{definition}\newtheorem{remarks}[theorem]{Remarks}
\newtheorem{notation}[theorem]{Notation}

%%%%%%%%%%%%%%%%%%%%%%%%%%%%%%%%%%%%%%%%%%%%%%%%%%%%%%%
% General Mathematics shorthands
%%%%%%%%%%%%%%%%%%%%%%%%%%%%%%%%%%%%%%%%%%%%%%%%%%%%%%%

% aliases
\newcommand{\infinity}{\infty}
\newcommand{\iso}{\cong}
\newcommand{\isomorphism}{\cong}
\newcommand{\vd}{\ensuremath{\vdash}}  % = turnstyle

\newcommand{\dom}{\operatorname{dom}}
\newcommand{\cod}{\operatorname{cod}}
\newcommand{\Tr}{\operatorname{Tr}}

\newcommand{\denote}[1]{% --``semantic'' brakets
\llbracket #1 \rrbracket} 
\newcommand{\ldenote}[1]{\left\llbracket #1 \right\rrbracket} 
\newcommand{\name}[1]{%---  name / coname 
\ulcorner #1 \urcorner}
\newcommand{\coname}[1]{%
\llcorner #1 \lrcorner}
\newcommand{\sizeof}[1]{% \sizeof{x} == |x|
  \left|#1\right|}


%%%%%%%%%%%%%%%%%%%%%%%%%%%%%%%%%%%%%%%%%%%%%%%%%%%%%%%
% Quantum Notations
%%%%%%%%%%%%%%%%%%%%%%%%%%%%%%%%%%%%%%%%%%%%%%%%%%%%%%%

\newcommand{\bra}[1]{%  dirac 'bra'
    \ensuremath{\left\langle #1 \right|}\xspace}
\newcommand{\ket}[1]{%  dirac 'ket'
    \ensuremath{\left|  #1 \right\rangle}\xspace}
\newcommand{\innp}[2]{%  dirac style inner product
    \ensuremath{\langle #1 \mid #2 \rangle}}
\newcommand{\outp}[2]{%  dirac style outer product
    \ensuremath{\left|\left. #1 \rangle\right.\langle\left. #2 \right|\right. }}
\newcommand{\norm}[1]{% define command to get spacing right
        \ensuremath{\left| #1 \right| }}

% aliases for circuits
\newcommand{\CZ}{\ensuremath{\wedge Z}\xspace}
\newcommand{\CX}{\ensuremath{\wedge X}\xspace}
\newcommand{\CNot}{\CX}
\newcommand{\CPhase}[1]{\ensuremath{\wedge R_Z(#1)}\xspace}


%%%%%%%%%%%%%%%%%%%%%%%%%%%%%%%%%%%%%%%%%%%%%%%%%%%%%%%
% Time to define some category  theory stuff
%%%%%%%%%%%%%%%%%%%%%%%%%%%%%%%%%%%%%%%%%%%%%%%%%%%%%%%

\usepackage{diagrams} % remember to give shout-out to Paul Taylor
\newarrow{Equals}{}{=}{}{=}{}

% Objects of ???
\newcommand{\OBJ}[1]{\ensuremath{\mathrm{Obj}_{#1}}}
% Objects of {\cal ??}
\newcommand{\OBJC}[1]{\ensuremath{\mathrm{Obj}_{{\cal #1}}}}
% Arrows of ???
\newcommand{\ARR}[1]{\ensuremath{\mathrm{Arr}_{#1}}}
% Arrows of {\cal ??}
\newcommand{\ARRC}[1]{\ensuremath{\mathrm{Arr}_{{\cal #1}}}}

% sub scripted identity morphisms
\ifx\numericids\undefined
\newcommand{\id}[1]{\ensuremath{\mathrm{id}_{#1}}}
\else
\newcommand{\id}[1]{\ensuremath{1_{#1}}}
\fi

% Caligraphic category names
\newcommand{\catA}{\ensuremath{{\cal A}}\xspace}
\newcommand{\catB}{\ensuremath{{\cal B}}\xspace}
\newcommand{\catC}{\ensuremath{{\cal C}}\xspace}
\newcommand{\catD}{\ensuremath{{\cal D}}\xspace}
\newcommand{\catE}{\ensuremath{{\cal E}}\xspace}
\newcommand{\catF}{\ensuremath{{\cal F}}\xspace}
\newcommand{\catG}{\ensuremath{{\cal G}}\xspace}
\newcommand{\catH}{\ensuremath{{\cal H}}\xspace}
\newcommand{\catP}{\ensuremath{{\cal P}}\xspace}
\newcommand{\catQ}{\ensuremath{{\cal Q}}\xspace}

% Boldified names of useful categories (and some less useful!)
%
\newcommand{\vectfd}[1]{% category of f.d. vector spaces over some field
\ensuremath{\mathbf{Vect}^{\mathrm{fd}}_{#1}}\xspace}
\newcommand{\vectfdc}{% category of f.d. vector spaces over complexes
\vectfd{\mathbb{C}}}
\newcommand{\catRel}{% the category of sets and relations
\ensuremath{\mathbf{Rel}}\xspace}
\newcommand{\catSet}{% the category of sets and functions
\ensuremath{\mathbf{Set}}\xspace}
\newcommand{\catCat}{% the category of (small) categories
\ensuremath{\mathbf{Cat}}\xspace}
\newcommand{\catInvCat}{% the category of involutive categories
\ensuremath{\mathbf{InvCat}}\xspace}
\newcommand{\catComCl}{% the category of compact closed  categories
\ensuremath{\mathbf{ComCl}}\xspace}
\newcommand{\catSComCl}{% the category of strongly compact closed  categories
\ensuremath{\mathbf{SComCl}}\xspace}
\newcommand{\catGrph}{% the category of graphs
\ensuremath{\mathbf{Grph}}\xspace}
\newcommand{\qubit}{% the category of qubits
\ensuremath{\mathbf{Qubit}}\xspace}
\newcommand{\fdhilb}{% the category of finite dimensional hilbert space
\ensuremath{\mathbf{fdHilb}}\xspace}
\newcommand{\fdHilb}{\fdhilb}
\newcommand{\catInvCom}{% the category of involutive categories
\ensuremath{\mathbf{InvCom}}\xspace}
\newcommand{\catCom}{% the category of compact closed  categories
\ensuremath{\mathbf{Com}}\xspace}
\newcommand{\superop}{% the category of superoperators
\ensuremath{\mathbf{SuperOp}}\xspace}

%%% I use this pretty often...
\newcommand{\zxcalculus}{\textsc{zx}-calculus\xspace}
\newcommand{\ZX}{\ensuremath{\mathbf{ZX}}\xspace}
\newcommand{\ZXs}{\ensuremath{\mathbf{ZX_s}}\xspace}

\newcommand{\Enc}{\ensuremath{\mathrm{Enc}}\xspace}
\newcommand{\Dec}{\ensuremath{\mathrm{Dec}}\xspace}

%%%%%%%% useful names for Cliffords

\usepackage{dsfont}
\newcommand{\ID}{\mathds{1}}

\newcommand{\Cliff}{\ensuremath{\mathbf{Cliff}}\xspace}

\newcommand{\C}[1]{%
\ensuremath{\mathcal{C}_{#1}\xspace}}

\newcommand{\CC}[1]{%
\ensuremath{\mathfrak{C}_{#1}\xspace}}

\renewcommand{\P}[1]{%
\ensuremath{\mathcal{P}_{#1}\xspace}}

\newcommand{\Circ}{%
\ensuremath{\mathbf{Circ}}\xspace}

\newcommand{\eq}{\ensuremath{\stackrel{*}{\leftrightarrow}}}
\newcommand{\eqqq}{\ensuremath{\stackrel{*}{\longleftrightarrow}}}
\newcommand{\rw}{\ensuremath{\stackrel{*}{\rightarrow}}}


\newcommand{\CNOT}{\ensuremath{\textsc{cnot}}\xspace}
\newcommand{\TONC}{\ensuremath{\textsc{tonc}}\xspace}





%%-----------------------------------------------------
%% TODO HERE --- TIKZ STUFF
%%-----------------------------------------------------

%\usepackage{pdflscape} % used to get somethings sideways


%%%% hyperref wants to be last
\usepackage{hyperref}
