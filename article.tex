\RequirePackage[l2tabu,orthodox]{nag}
\documentclass[a4paper]{article}
\bibliographystyle{plain}

\pdfoutput=1

\usepackage{rwd-drafting}

%
% LaTeX2e Preamble / Ross Duncan 2011
%
%-------------------------------------------------------
% TODO 27/1/2011
% 1. Something for tikz
% 2. Font alternatives?
% 3. Which journals clash with which bits?
% 4. No support for logic -- proof trees linear logic symbols etc
% 5. Use switches to turn on/off some parts
%-------------------------------------------------------

% typographical improvements
\usepackage[T1]{fontenc}
\usepackage{lmodern}
\usepackage{fixltx2e}
\usepackage{microtype} 
%\usepackage[stretch=10]{microtype} % more conservative
%\usepackage[expansion=false]{microtype} % still more conservative
\usepackage{xspace}

% make figs a bit nicer
\newcommand{\figureline}{\rule{\textwidth}{0.5pt}}
\newcommand{\figureend}{\rule{\textwidth}{0.5pt}}

% do The Right Thing
\makeatletter
\newcommand\etc{etc\@ifnextchar.{}{.\@}\xspace}
\newcommand\ie{i.e.\@\xspace}  % these two may be broken
\newcommand\eg{e.g.\@\xspace}
\makeatother

% arg is height; optional arg is baseline shift
\newcommand{\vstrut}[2][0pt]{\rule[#1]{0pt}{#2}}

%%%%%%%%%%%%%%%%%%%%%%%%%%%%%%%%%%%%%%%%%%%%%%%%%%%%%%%
% Include graphics in a nice way
%%%%%%%%%%%%%%%%%%%%%%%%%%%%%%%%%%%%%%%%%%%%%%%%%%%%%%%
\usepackage{graphicx}

%%%% Use these commands to insert graphics into 
%%%% the text and have them line up properly.

%%% - inlinegraphic requires absolute size to be given
%%% - ninlinegraphic uses optional scaling factor

%%%% first param is the desired height of the graphic
%%%% if it is less the 1ex, it might not look good
\newcommand{\inlinegraphic}[2]{
  %% todo -- make this thing calculate the height 
  %% itself based on a global scaling factor
  \dimendef\grafheight=255\dimendef\grafvshift=254
  \grafheight=#1
  \grafvshift=-0.5\grafheight
  \advance\grafvshift by 0.5ex
  \raisebox{\grafvshift}{\includegraphics[height=\grafheight]{images/#2}\xspace}
}


%%% optional first arg is scaling factor
\newcommand{\ninlinegraphic}[2][1.0]{
  \dimendef\grafheight=255\dimendef\grafvshift=254
  \setbox0 = \hbox{\scalebox{#1}{\includegraphics{images/#2}}}
  \grafheight=\the\ht0
  \grafvshift=-0.5\grafheight
  \advance\grafvshift by 0.5ex
  \raisebox{\grafvshift}{\includegraphics[height=\grafheight]{images/#2}\xspace}
}

\newcommand{\inline}[1]{
  \raisebox{0.5ex}{\;#1\;}
}


%%%%%%%%%%%%%%%%%%%%%%%%%%%%%%%%%%%%%%%%%%%%%%%%%%%%%%%
%% Mathematical packages and theorems
%%%%%%%%%%%%%%%%%%%%%%%%%%%%%%%%%%%%%%%%%%%%%%%%%%%%%%%

\usepackage{latexsym}
\usepackage{amssymb}
\usepackage{amsmath} % can collide with some journals
\usepackage{stmaryrd}
\usepackage{mathtools}
%%% not needed if amsmath loaded
% \newcommand{\text}[1]{\ensuremath{\mbox{#1}}}


%%% Defining theorems and friends usually makes problems for journal
%%% styles
\usepackage{amsthm}
\newtheorem{theorem}{Theorem}[section]
\newtheorem{proposition}[theorem]{Proposition}
\newtheorem{lemma}[theorem]{Lemma}
\newtheorem{corollary}[theorem]{Corollary}
\theoremstyle{definition}\newtheorem{example}[theorem]{Example}
\theoremstyle{definition}\newtheorem{examples}[theorem]{Examples}
\theoremstyle{definition}\newtheorem{definition}[theorem]{Definition}
\theoremstyle{definition}\newtheorem{definitions}[theorem]{Definitions}
\theoremstyle{definition}\newtheorem{remark}[theorem]{Remark}
\theoremstyle{definition}\newtheorem{remarks}[theorem]{Remarks}
\newtheorem{notation}[theorem]{Notation}

%%%%%%%%%%%%%%%%%%%%%%%%%%%%%%%%%%%%%%%%%%%%%%%%%%%%%%%
% General Mathematics shorthands
%%%%%%%%%%%%%%%%%%%%%%%%%%%%%%%%%%%%%%%%%%%%%%%%%%%%%%%

% aliases
\newcommand{\infinity}{\infty}
\newcommand{\iso}{\cong}
\newcommand{\isomorphism}{\cong}
\newcommand{\vd}{\ensuremath{\vdash}}  % = turnstyle

\newcommand{\dom}{\operatorname{dom}}
\newcommand{\cod}{\operatorname{cod}}
\newcommand{\Tr}{\operatorname{Tr}}

\newcommand{\denote}[1]{% --``semantic'' brakets
\llbracket #1 \rrbracket} 
\newcommand{\ldenote}[1]{\left\llbracket #1 \right\rrbracket} 
\newcommand{\name}[1]{%---  name / coname 
\ulcorner #1 \urcorner}
\newcommand{\coname}[1]{%
\llcorner #1 \lrcorner}
\newcommand{\sizeof}[1]{% \sizeof{x} == |x|
  \left|#1\right|}


%%%%%%%%%%%%%%%%%%%%%%%%%%%%%%%%%%%%%%%%%%%%%%%%%%%%%%%
% Quantum Notations
%%%%%%%%%%%%%%%%%%%%%%%%%%%%%%%%%%%%%%%%%%%%%%%%%%%%%%%

\newcommand{\bra}[1]{%  dirac 'bra'
    \ensuremath{\left\langle #1 \right|}\xspace}
\newcommand{\ket}[1]{%  dirac 'ket'
    \ensuremath{\left|  #1 \right\rangle}\xspace}
\newcommand{\innp}[2]{%  dirac style inner product
    \ensuremath{\langle #1 \mid #2 \rangle}}
\newcommand{\outp}[2]{%  dirac style outer product
    \ensuremath{\left|\left. #1 \rangle\right.\langle\left. #2 \right|\right. }}
\newcommand{\norm}[1]{% define command to get spacing right
        \ensuremath{\left| #1 \right| }}

% aliases for circuits
\newcommand{\CZ}{\ensuremath{\wedge Z}\xspace}
\newcommand{\CX}{\ensuremath{\wedge X}\xspace}
\newcommand{\CNot}{\CX}



%%%%%%%%%%%%%%%%%%%%%%%%%%%%%%%%%%%%%%%%%%%%%%%%%%%%%%%
% Time to define some category  theory stuff
%%%%%%%%%%%%%%%%%%%%%%%%%%%%%%%%%%%%%%%%%%%%%%%%%%%%%%%

\usepackage{diagrams} % remember to give shout-out to Paul Taylor
\newarrow{Equals}{}{=}{}{=}{}

% Objects of ???
\newcommand{\OBJ}[1]{\ensuremath{\mathrm{Obj}_{#1}}}
% Objects of {\cal ??}
\newcommand{\OBJC}[1]{\ensuremath{\mathrm{Obj}_{{\cal #1}}}}
% Arrows of ???
\newcommand{\ARR}[1]{\ensuremath{\mathrm{Arr}_{#1}}}
% Arrows of {\cal ??}
\newcommand{\ARRC}[1]{\ensuremath{\mathrm{Arr}_{{\cal #1}}}}

% sub scripted identity morphisms
\ifx\numericids\undefined
\newcommand{\id}[1]{\ensuremath{\mathrm{id}_{#1}}}
\else
\newcommand{\id}[1]{\ensuremath{1_{#1}}}
\fi

% Caligraphic category names
\newcommand{\catA}{\ensuremath{{\cal A}}\xspace}
\newcommand{\catB}{\ensuremath{{\cal B}}\xspace}
\newcommand{\catC}{\ensuremath{{\cal C}}\xspace}
\newcommand{\catD}{\ensuremath{{\cal D}}\xspace}
\newcommand{\catE}{\ensuremath{{\cal E}}\xspace}
\newcommand{\catF}{\ensuremath{{\cal F}}\xspace}
\newcommand{\catG}{\ensuremath{{\cal G}}\xspace}
\newcommand{\catH}{\ensuremath{{\cal H}}\xspace}
\newcommand{\catP}{\ensuremath{{\cal P}}\xspace}
\newcommand{\catQ}{\ensuremath{{\cal Q}}\xspace}

% Boldified names of useful categories (and some less useful!)
%
\newcommand{\vectfd}[1]{% category of f.d. vector spaces over some field
\ensuremath{\mathbf{Vect}^{\mathrm{fd}}_{#1}}\xspace}
\newcommand{\vectfdc}{% category of f.d. vector spaces over complexes
\vectfd{\mathbb{C}}}
\newcommand{\catRel}{% the category of sets and relations
\ensuremath{\mathbf{Rel}}\xspace}
\newcommand{\catSet}{% the category of sets and functions
\ensuremath{\mathbf{Set}}\xspace}
\newcommand{\catCat}{% the category of (small) categories
\ensuremath{\mathbf{Cat}}\xspace}
\newcommand{\catInvCat}{% the category of involutive categories
\ensuremath{\mathbf{InvCat}}\xspace}
\newcommand{\catComCl}{% the category of compact closed  categories
\ensuremath{\mathbf{ComCl}}\xspace}
\newcommand{\catSComCl}{% the category of strongly compact closed  categories
\ensuremath{\mathbf{SComCl}}\xspace}
\newcommand{\catGrph}{% the category of graphs
\ensuremath{\mathbf{Grph}}\xspace}
\newcommand{\qubit}{% the category of qubits
\ensuremath{\mathbf{Qubit}}\xspace}
\newcommand{\fdhilb}{% the category of finite dimensional hilbert space
\ensuremath{\mathbf{fdHilb}}\xspace}
\newcommand{\fdHilb}{\fdhilb}
\newcommand{\catInvCom}{% the category of involutive categories
\ensuremath{\mathbf{InvCom}}\xspace}
\newcommand{\catCom}{% the category of compact closed  categories
\ensuremath{\mathbf{Com}}\xspace}
\newcommand{\superop}{% the category of superoperators
\ensuremath{\mathbf{SuperOp}}\xspace}

%%% I use this pretty often...
\newcommand{\zxcalculus}{\textsc{zx}-calculus\xspace}
\newcommand{\ZX}{\ensuremath{\mathbf{ZX}}\xspace}
\newcommand{\ZXs}{\ensuremath{\mathbf{ZX_s}}\xspace}

\newcommand{\Enc}{\ensuremath{\mathrm{Enc}}\xspace}
\newcommand{\Dec}{\ensuremath{\mathrm{Dec}}\xspace}

%%%%%%%% useful names for Cliffords

\usepackage{dsfont}
\newcommand{\ID}{\mathds{1}}

\newcommand{\Cliff}{\ensuremath{\mathbf{Cliff}}\xspace}

\newcommand{\C}[1]{%
\ensuremath{\mathcal{C}_{#1}\xspace}}

\newcommand{\CC}[1]{%
\ensuremath{\mathfrak{C}_{#1}\xspace}}

\renewcommand{\P}[1]{%
\ensuremath{\mathcal{P}_{#1}\xspace}}

\newcommand{\eq}{\ensuremath{\stackrel{*}{\leftrightarrow}}}
\newcommand{\eqqq}{\ensuremath{\stackrel{*}{\longleftrightarrow}}}
\newcommand{\rw}{\ensuremath{\stackrel{*}{\rightarrow}}}


\newcommand{\CNOT}{\ensuremath{\textsc{cnot}}\xspace}
\newcommand{\TONC}{\ensuremath{\textsc{tonc}}\xspace}




%%-----------------------------------------------------
%% TODO HERE --- TIKZ STUFF
%%-----------------------------------------------------

%\usepackage{pdflscape} % used to get somethings sideways


%%%% hyperref wants to be last
\usepackage{hyperref}

\usepackage{tikz}
\usepackage{pgfplots}
\usetikzlibrary{trees}
\usetikzlibrary{topaths}
\usetikzlibrary{decorations.pathmorphing}
\usetikzlibrary{decorations.markings}
\usetikzlibrary{matrix,backgrounds,folding}
\usetikzlibrary{chains,scopes,positioning,fit}
\usetikzlibrary{arrows,shadows}
\usetikzlibrary{calc} 
\usetikzlibrary{chains}
\usetikzlibrary{shapes,shapes.geometric,shapes.misc}

\usepackage[undirected]{tikzquanto}
\usepackage{tikzcircuits}

% \newcommand{\tikzfig}[1]{\inline{%
% \beginpgfgraphicnamed{#1}
% \InputIfFileExists{#1.tikz}{}{\input{./tikz/#1.tikz}}
% \endpgfgraphicnamed}}

% \newcommand{\ctikzfig}[1]{%
% \begin{center}\rm
%   \tikzfig{#1}
% \end{center}}

%\newcommand{\inlinetikzfig}[1]{\InputIfFileExists{#1.tikz}{}{\input{./tikz/#1.tikz}}}

\pgfdeclarelayer{edgelayer}
\pgfdeclarelayer{nodelayer}
\pgfsetlayers{background,edgelayer,nodelayer,main}

% this is the "quanto" style
\tikzstyle{every picture}=[baseline=(current bounding box).east,node distance=5mm]
\tikzstyle{none}=[inner sep=0mm]
\tikzstyle{every loop}=[]

% this is a work around for tikzit bug
\tikzstyle{(null)}=[]
\tikzstyle{plain}=[]

%\tikzstyle{every loop}=[]

% \tikzstyle{cloud} = [draw=gray!18,fill=gray!10, ellipse, node
% distance=3cm,minimum height=3em, minimum width=5em]

% some predefined tikz drawings
%\input{tikz/gens.tex}



%%% This by Liam -- think it's not needed.
\tikzset{
operator/.style = {draw,fill=white,minimum size=1.5em},%
operator2/.style ={draw,fill=white,minimum height=7cm,minimum width=1.5 cm},%
phase/.style = {draw,fill,shape=circle,minimum size=5pt,inner sep=0pt},% 
endline/.style = {draw,fill,shape=circle,minimum size=1pt,inner sep=0pt},%
surround/.style = {fill=blue!10,thick,draw=black,rounded corners=2mm},%
cross/.style={},
    % path picture={ \draw[thick,black](path picture bounding%
    % box.north) -- (path picture bounding box.south) (path picture%
    % bounding box.west) -- (path picture bounding box.east);}},%
circlewc/.style={draw,circle,cross,minimum width=0.3 cm},%
} %% end tikzset

%%% Normal circuits vvv

\newcommand{\circbox}[1]{\inline{%
\begin{tikzpicture}[circuit]
  \node  (0) at (1,0) {};
  \node [box] (1) at (0, 0) {$#1$};
  \node  (2) at (-1,0) {};
  \draw  (0) to (1);
  \draw  (1) to (2);
\end{tikzpicture}
}}

\newcommand{\circX}[1]{\circbox{X_#1}}
\newcommand{\circZ}[1]{\circbox{Z_#1}}
\newcommand{\circH}{\circbox{H_{}}}
\newcommand{\circS}{\circbox{S_{}}}
\newcommand{\circV}{\circbox{V_{}}}



\newcommand{\circket}[1]{\inline{%
\begin{tikzpicture}[circuit]
  \node [box] (1) at (0, 0) {$\ket{#1}$};
  \node  (2) at (0, -1.5) {};
  \draw  (1) to (2);
\end{tikzpicture}
}}

\newcommand{\circketplus}{\circket{\!+\!}}

\newcommand{\circketplusDef}{\inline{%
\begin{tikzpicture}[circuit]
  \node [keto box] (0) at (0, 1.5) {};
  \node [H box] (1) at (0, 0) {};
  \node  (2) at (0, -1.25) {};
  \draw  (0) to (1);
  \draw  (1) to (2);
\end{tikzpicture}
}}

\newcommand{\circCX}{\inline{%
\begin{tikzpicture}[circuit]
  \node  (0) at (1, 0) {};
  \node  (1) at (1, 1) {};
% \node [style=ctrl vertex] (2) at (-1, 0) {};
% \node [style=target vertex] (3) at (1, 0) {};
  \circcnot{3}{0,1}{2}{0,0}
  \node  (4) at (-1, 0) {};
  \node  (5) at (-1,1) {};
  \draw  (1) to (3);
  \draw  (0) to (2);
  \draw  (3) to (5);
  \draw  (2) to (4);
\end{tikzpicture}
}}

\newcommand{\circCZgate}{\inline{%
\begin{tikzpicture}[circuit]
  \node  (0) at (1, 0) {};
  \node  (1) at (1, 1) {};
% \node [style=ctrl vertex] (2) at (-1, 0) {};
% \node [style=target vertex] (3) at (1, 0) {};
  \circczed{3}{0,1}{2}{0, 0}
  \node  (4) at (-1, 0) {};
  \node  (5) at (-1, 1) {};
  \draw  (1) to (3);
  \draw  (0) to (2);
  \draw  (3) to (5);
  \draw  (2) to (4);
\end{tikzpicture}
}}

\newcommand{\circCPhase}[1]{\inline{%
\begin{tikzpicture}[circuit]
  \node  (0) at (1, 0) {};
  \node  (1) at (1, 1) {};
% \node [style=ctrl vertex] (2) at (-1, 0) {};
% \node [style=target vertex] (3) at (1, 0) {};
  \circcphase{3}{0,1}{2}{0, 0}{#1}
  \node  (4) at (-1, 0) {};
  \node  (5) at (-1, 1) {};
  \draw  (1) to (3);
  \draw  (0) to (2);
  \draw  (3) to (5);
  \draw  (2) to (4);
\end{tikzpicture}
}}


\newcommand{\circCZcirc}{\inline{%
\begin{tikzpicture}[circuit]
  \node  (0) at (-1, 2.5) {};
  \node  (1) at (1, 2.5) {};
  \node [style=H box] (2) at (1, 1.25) {};
% \node [style=ctrl vertex] (3) at (-1, 0) {};
% \node [style=target vertex] (4) at (1, 0) {};
  \circcnot{3}{-1, 0}{4}{1, 0}
  \node [style=H box] (5) at (1, -1.25) {};
  \node  (6) at (-1, -2.5) {};
  \node  (7) at (1, -2.5) {};
  \draw  (5) to (7);
  \draw  (3) to (6);
  \draw  (4) to (5);
  \draw  (1) to (2);
  \draw  (2) to (4);
  \draw  (0) to (3);
\end{tikzpicture}
}}


\newcommand{\circMakeBell}{\inline{%
\begin{tikzpicture}[circuit]
  \node [style=keto box] (0) at (-1, 2.5) {};
  \node [style=keto box] (1) at (1, 2.5) {};
  \node [style=H box] (2) at (-1, 1.125) {};
% \node [style=ctrl vertex] (3) at (-1, 0) {};
% \node [style=target vertex] (4) at (1, 0) {};
  \circcnot{3}{-1, 0}{4}{1, 0}
  \node  (5) at (-1, -1) {};
  \node  (6) at (1, -1) {};
  \draw  (2) to (3);
  \draw  (4) to (6);
  \draw  (1) to (4);
  \draw  (0) to (2);
  \draw  (3) to (5);
\end{tikzpicture}
}}

\newcommand{\circMakeCluster}{\inline{%
\begin{tikzpicture}[quanto]
  \node  (0) at (-4, 3) {};
  \node [ketp box] (1) at (-2, 3.5) {};
  \node [ketp box] (2) at (0, 3.5) {};
  \node [ketp box] (3) at (2, 3.5) {};
  \node [ketp box] (4) at (4, 3.5) {};
  \node  (5) at (6, 3) {};
  % \node [ctrl vertex] (6) at (-4, 2) {};
  % \node [target vertex] (7) at (-2, 2) {};
  \circczed{6}{-4, 2}{7}{-2, 2}
  % \node [ctrl vertex] (8) at (0, 2) {};
  % \node [target vertex] (9) at (2, 2) {};
  \circczed{8}{0, 2}{9}{2, 2}
  % \node [ctrl vertex] (10) at (4, 2) {};
  % \node [target vertex] (11) at (6, 2) {};
  \circczed{10}{4, 2}{11}{6, 2}
  \node  (12) at (-4, 1) {};
  % \node [ctrl vertex] (13) at (-2, 1) {};
  % \node [target vertex] (14) at (0, 1) {};
  \circczed{13}{-2, 1}{14}{0, 1}
  % \node [ctrl vertex] (15) at (2, 1) {};
  % \node [target vertex] (16) at (4, 1) {};
  \circczed{15}{2, 1}{16}{4, 1}
  \node  (17) at (6, 1) {};
  \node  (18) at (-2, 0) {};
  \node  (19) at (0, 0) {};
  \node  (20) at (2, 0) {};
  \node  (21) at (4, 0) {};
  \draw  (13) to (18);
  \draw  (8) to (14);
  \draw  (11) to (17);
  \draw  (6) to (12);
  \draw  (16) to (21);
  \draw  (4) to (10);
  \draw  (1) to (7);
  \draw  (9) to (15);
  \draw  (5) to (11);
  \draw  (14) to (19);
  \draw  (15) to (20);
  \draw  (7) to (13);
  \draw  (0) to (6);
  \draw  (2) to (8);
  \draw  (10) to (16);
  \draw  (3) to (9);
  %%% added by hand
  \node [ellipses] at (-4,3.5) {};
  \node [ellipses] at (6,3.5) {};
  \node [ellipses] at (-4,0.5) {};
  \node [ellipses] at (6,0.5) {};
\end{tikzpicture}
}}

\newcommand{\circmeas}{\inline{%
\begin{tikzpicture}[circuit]
  \node [Ma box] (0) at (0, -1.5) {};
  \node (1) at (0, 0) {};
  \draw  (0) to (1);
\end{tikzpicture}
}}

\newcommand{\circClassMeas}{\inline{%
\begin{tikzpicture}[circuit]
  \node [Ma box] (0) at (0, -1.5) {};
  \node (1) at (0, 0) {};
  \node (2) at (0,-3) {};
  \draw  (0) to (1);
  \draw [double,thin] (0) to (2) ;
\end{tikzpicture}
}}

\newcommand{\circClassPauliX}{\inline{%
\begin{tikzpicture}[circuit]
  \node  (0) at (-1.5, 1.5) {};
  \node  (1) at (0, 1.5) {};
  \node  (2) at (-1.5, 0) {};
  \node [Xp box] (3) at (0, 0) {};
  \node  (4) at (0, -1.5) {};
  \draw  [double,thin] (2.center) to (3);
  \draw  [double,thin] (0) to (2.center);
  \draw  (3) to (4);
  \draw  (1) to (3);
\end{tikzpicture}
}}

\newcommand{\circClassPauliZ}{\inline{%
\begin{tikzpicture}[circuit]
  \node  (0) at (-1.5, 1.5) {};
  \node  (1) at (0, 1.5) {};
  \node  (2) at (-1.5, 0) {};
  \node [Zp box] (3) at (0, 0) {};
  \node  (4) at (0, -1.5) {};
  \draw  [double,thin] (2.center) to (3);
  \draw  [double,thin] (0) to (2.center);
  \draw  (3) to (4);
  \draw  (1) to (3);
\end{tikzpicture}
}}

\newcommand{\circDelta}{\inline{%
\begin{tikzpicture}[circuit]
  \node  (0) at (0, 1) {};
  \node [boundary vertex] (1) at (0, 0) {};
  \node  (2) at (-1, -1) {};
  \node  (3) at (1, -1) {};
  \draw [double,thin] (3.center) to (1);
  \draw [double,thin] (1) to (0.center);
  \draw [double,thin] (2.center) to (1);
\end{tikzpicture}
}}

\newcommand{\circEpsilon}{\inline{%
\begin{tikzpicture}[circuit]
  \node  (0) at (0, 1) {};
  \node [boundary vertex] (1) at (0, 0) {};
  \draw [double,thin] (1) to (0.center);
\end{tikzpicture}
}}

\newcommand{\circDeltaLeft}{\inline{%
\begin{tikzpicture}[circuit]
  \node  (0) at (0, 1) {};
  \node [style=boundary vertex] (1) at (0, 0) {};
  \node [style=boundary vertex] (2) at (-1, -1) {};
  \node  (3) at (-2, -2) {};
  \node  (4) at (0, -2) {};
  \node  (5) at (2, -2) {};
  \draw [double,thin] (2) to (3.center);
  \draw [double,thin] (2) to (4.center);
  \draw [double,thin] (1) to (2);
  \draw [double,thin] (1) to (5.center);
  \draw [double,thin] (1) to (0.center);
\end{tikzpicture}
}}

\newcommand{\circDeltaRight}{\inline{%
\begin{tikzpicture}[circuit]
  \node  (0) at (0, 1) {};
  \node [style=boundary vertex] (1) at (0, 0) {};
  \node [style=boundary vertex] (2) at (1, -1) {};
  \node  (3) at (-2, -2) {};
  \node  (4) at (0, -2) {};
  \node  (5) at (2, -2) {};
  \draw [double,thin] (2) to (5.center);
  \draw [double,thin] (2) to (4.center);
  \draw [double,thin] (1) to (2);
  \draw [double,thin] (1) to (3.center);
  \draw [double,thin] (1) to (0.center);
\end{tikzpicture}
}}

\newcommand{\circDeltaUnit}{\inline{%
\begin{tikzpicture}[circuit]
  \node  (0) at (0, 1) {};
  \node [style=boundary vertex] (1) at (0, 0) {};
  \node [style=boundary vertex] (2) at (-1, -1) {};
  \node  (3) at (1, -1) {};
  \draw [double,thin] (1) to (2);
  \draw [double,thin] (1) to (3.center);
  \draw [double,thin] (1) to (0.center);
\end{tikzpicture}
}}

\newcommand{\circDeltaSigma}{\inline{%
\begin{tikzpicture}[circuit]
  \node (0) at (0, 1) {};
  \node [style=boundary vertex] (1) at (0, 0) {};
  \node  (2) at (-1, -1) {};
  \node  (3) at (1, -1) {};
  \node  (4) at (-1, -2.5) {};
  \node  (5) at (1, -2.5) {};
  \draw [double,thin] (1) to (0.center);
  \draw [double,thin][out=90, looseness=0.75, in=-90] (5.center) to (2.center);
  \draw [double,thin][out=90, looseness=0.75, in=-90] (4.center) to (3.center);
  \draw [double,thin] (1) to (2.center);
  \draw [double,thin] (1) to (3.center);
\end{tikzpicture}
}}

\newcommand{\circId}{\inline{%
\begin{tikzpicture}[circuit]
  \node  (0) at (0, 1) {};
  \node  (1) at (0, -1) {};
  \draw [double,thin] (1.center) to (0.center);
\end{tikzpicture}
}}

\newcommand{\circMeasErase}{\inline{%
\begin{tikzpicture}[circuit]
  \node  (0) at (0, 1.5) {};
  \node [style=Ma box] (1) at (0, 0) {};
  \node [style=boundary vertex] (2) at (0, -1.5) {};
  \draw [double,thin] (1) to (2);
  \draw  (0.center) to (1);
\end{tikzpicture}
}}

\newcommand{\circTeleport}{\inline{%
\begin{tikzpicture}[circuit]
  \node [label=above:Alice] (0) at (-2, 5) {};
  \node [keto box] (1) at (0, 5) {};
  \node [keto box] (2) at (2, 5) {};
  \node [H box] (3) at (0, 3.5) {};
  \node [ctrl vertex] (4) at (0, 2) {};
  \node [target vertex] (5) at (2, 2) {};
  \node [target vertex] (6) at (-2, 0.5) {};
  \node [ctrl vertex] (7) at (0, 0.5) {};
  \draw (4) to (5);
  \draw (6) to (7);
  \node [H box] (8) at (-2, -1) {};
  \node [Mo box] (9) at (0, -1) {};
  \node [Mo box] (10) at (-2, -2.5) {};
  \node  (11) at (0, -2.5) {};
  \node [Zp box] (12) at (2, -2.5) {};
  \node  (13) at (-2, -4) {};
  \node [Xp box] (14) at (2, -4) {};
  \node [label=below:Bob] (15) at (2, -5) {};
  \draw  (4) to (7);
  \draw  (3) to (4);
  \draw  [double,thin] (10) to (13.center);
  \draw  (0.center) to (6);
  \draw  (6) to (8);
  \draw  [double,thin] (9) to (11.center);
  \draw  (8) to (10);
  \draw  (12) to (14);
  \draw  (14) to (15.center);
  \draw  (1) to (3);
  \draw  (2) to (5);
  \draw  [double,thin] (13.center) to (14);
  \draw  [double,thin] (11.center) to (12);
  \draw  (7) to (9);
  \draw  (5) to (12);
  \begin{pgfonlayer}{background}
    \node [draw=gray!40,fill=gray!30,inner ysep=-9mm,inner
    xsep=-7.5mm,fit=(1) (2) (4) (5),label={[gray!85]right:B}] {}; 
    \node [draw=gray!40,fill=gray!30,inner ysep=-9mm,inner
    xsep=-7.5mm,fit=(9) (10) (6) (7),label={[gray!85]left:A}] {};
  \end{pgfonlayer}
\end{tikzpicture}
}}



%%% Local Variables: 
%%% mode: latex
%%% TeX-master: t
%%% End: 





\title{A Resource Theory of Quantum Circuit Rewriting}
\date{}
\author{Ross Duncan \and Denis Rochette}




\begin{document}
% For article.cls abstract goes here 
% other classes may have it in front-matter
\maketitle
\begin{abstract}
We formalise an abstract circuit rewriting system as a 2-PROP and,
given a valuation of the 1-cells in a suitable quantale $\mathbb{Q}$,
we show that the rewriting system is resource theory over
$\mathbb{Q}$.  We present some examples of this framework applied to
quantum circuit optimisation.
\end{abstract}



\section{Introduction}
\label{sec:intro}

\section{Circuits and 2-PROPS}
\label{sec:prelim}

We assume that the reader is familiar with the notion of symmetric
monoidal category; a standard reference is
\cite{MacLane:CatsWM:1971}.  

\begin{definition} \label{def:PROP}
  A \emph{PROP} is strict symmetric monoidal category whose objects
  are the natural numbers, and where the action of the tensor product on
  objects is given by addition.
\end{definition}

PROPs have been used to formalise many algebraic structures
\cite{Lack:2004sf}, however our interest here is to formalise a
general notion of \emph{circuit} made by wiring together certain
atomic operations.  With this in mind, a PROP of circuits is
henceforth a PROP freely generated from some monoidal signature; the
members of the signature are called \emph{gates}, and a circuit is
a morphism in such a PROP.

\begin{example} \label{ex:boolean-circs}
  TODO -- boolean circuits
\end{example}

\begin{example} \label{ex:quantum-circs}
  The main example of interest is \Circ, the PROP of quantum circuits.
  Different authors, and different hardware manufacturers, prefer
  different primitive gate sets but the following is typical:
  \[
  \begin{array}{cccccccccc}
    \circZ{{}} & \qquad  & \circX{{}} & \qquad & \circH \\ 
    \vstrut{1.2em}
    Z : 1 \to 1 && X : 1 \to 1 && H : 1 \to 1\\
    \\
    \circbox{R_Z(\theta)} & \qquad & \circCPhase{\theta} & \qquad & \circCX  \\ 
    \vstrut{1.2em}
    R_Z(\theta) : 1 \to 1 &&  \CPhase\theta :2 \to 2 && \CX : 2 \to 2\\
  \end{array}
  \]
  We omit the usual matrix interpretations but these can be found in
  the standard textbook \cite{NieChu:QuantComp:2000}.  Note that in
  this example we actually have an uncountable family of generating
  gates, one for each value $\theta \in [0,2\pi)$.  Unlike the case of
  Boolean circuits all quantum gates have the same number of inputs and
  outputs.  In fact, \Circ is a \dag-PROP and all its but this will not concern
  us here.
\end{example}

\begin{example}  \label{ex:zx-calculus}
  TODO -- zx-calculus
\end{example}

\noindent
Note that all of these examples usually include an equational theory.
We will don't include these equations in the  definition since we will
model them as higher dimensional arrows.

%%%%%%%%%%%%%%%%%%%%%%%%%%%%%%%%%%%%%%%%%
%%        2-CATEGORY DEFINITION        %%
%%%%%%%%%%%%%%%%%%%%%%%%%%%%%%%%%%%%%%%%%

\begin{definition}  \label{def:strict-two-cat}
  A \emph{strict} 2-category $\mathcal{C}$ is a category \emph{enriched over
    \textbf{Cat}}, the category of categories, \emph{viz:}
  \begin{itemize}
    \item A category $\mathcal{C}$, whose objects and arrows are known as
      \emph{0-cells} and \emph{1-cells} respectively.
    \item For every pair of objects $A$ and $B$, we have a
      category $\mathcal{C}(A, B)$, whose arrows are known as \emph{2-cells}.
    \item For \emph{2-cells} $\alpha : f \Rightarrow g$ and $\beta : g
      \Rightarrow h$ in $\mathcal{C}(A,B)$ their (vertical) composite is denoted
      $\beta \bullet \alpha : f \Rightarrow h$.
    \item For every 1-cell $f:A\to B$ we denote its identity 2-cell by
      $Id_f:f \Rightarrow f$.
    \item For all \emph{0-cells} $A$, $B$, $C$ a bifunctor $\circ:
      \mathcal{C}(B, C) \times \mathcal{C}(A, B) \to \mathcal{C}(A, C)$ called
      \emph{horizontal composition} of \emph{2-cells}.
  \end{itemize}
\REM{Need here the name for a 2-cat with all \emph{2-cells} invertible;  the
  term 2-groupoid refers to the case where both \emph{1-} and \emph{2-cells} are
  invertible.}
\end{definition}

\noindent
In the following we'll always write $A, B, C$ for \emph{0-cells}, $f,
g, h$ for \emph{1-cells} and $\alpha$, $\beta$, $\gamma$ for
\emph{2-cells}.

%%%%%%%%%%%%%%%%%%%%%%%%%%%%%%%%%%%%%%%%
%%        2-FUNCTOR DEFINITION        %%
%%%%%%%%%%%%%%%%%%%%%%%%%%%%%%%%%%%%%%%%
\begin{definition} \label{def:two-functor}
  A \emph{strict} 2-functor $F$ between two \emph{strict} 2-categories
    $\mathcal{C}$ and $\mathcal{D}$ is
  \begin{itemize}
    \item for all \emph{0-cells} $A \in \mathcal{C}$ a 0-cell $F(A) \in
      \mathcal{D}$
    \item for all \emph{0-cells} $(A, B) \in \mathcal{C}$ a functor $F_{A, B}:
      \mathcal{C}(A, B) \to \mathcal{D}(F(A), F(B))$
  \end{itemize}

  \noindent
  Such that $F$ respect the horizontal and vertical compositions.
\end{definition}

%%%%%%%%%%%%%%%%%%%%%%%%%%%%%%%%%%%%%%%%%%%%%%%%%%
%%        MONOIDAL 2-CATEGORY DEFINITION        %%
%%%%%%%%%%%%%%%%%%%%%%%%%%%%%%%%%%%%%%%%%%%%%%%%%%
\begin{definition}\label{def:monoidal-two-cat}
  A \emph{strict} monoidal 2-category $(\mathcal{C}, \otimes, I)$ is a
  strict 2-category with a 2-bifunctor $\otimes: \mathcal{C} \times
  \mathcal{C} \to \mathcal{C}$ such that
  \begin{itemize}
    \item for all \emph{0-cells} $A$
      \[
        A \otimes I = A
      \]
    \item for all \emph{0-cells} $(A, B)$, $I$ is the monoidal unit of
    $\mathcal{C}(A, B)$
  \end{itemize}
  \TODO{Denis: fix this definition}
  \REM{What about the unit object?}
\end{definition}


%%%%%%%%%%%%%%%%%%%%%%%%%%%%%%%%%%%%%%%%%%
%%        INTERCHANGE LAW REMARK        %%
%%%%%%%%%%%%%%%%%%%%%%%%%%%%%%%%%%%%%%%%%%
\begin{remark}[Interchange law]
\TODO{DENIS : please fix this statement.  THe second and third clauses
  are the same.!}
  Since $\circ$ and $\otimes$ are bifunctorial, we have the following
    interchange equations for all \emph{0-cells} $A, B \in \mathcal{C}$ and all
    \emph{2-cells} $\alpha, \beta, \gamma, \delta \in \mathcal{C}(A, B)$
  \begin{align}
    (\alpha \circ \beta) \bullet (\gamma \circ \delta) &= (\alpha \bullet
      \gamma) \circ (\beta \bullet \delta) \\
    (\alpha \otimes \beta) \bullet (\gamma \otimes \delta) &= (\alpha \bullet
      \gamma) \otimes (\beta \bullet \delta)
  \end{align}
  Since $\otimes$ is a 2-functor, we have for all \emph{0-cells} $A, B \in
    \mathcal{C}$ and all \emph{2-cells} $\alpha, \beta, \gamma, \delta \in
    \mathcal{C}(A, B)$
  \[
    (\alpha \otimes \beta) \circ (\gamma \otimes \delta) = (\alpha \circ \gamma)
      \otimes (\beta \circ \delta)
  \]
\end{remark}

Putting the preceding definitions together, a 2-PROP is then a strict
monoidal 2-category whose underlying monoidal 1-category is a PROP, or
equivalently, a strict monoidal 2-category whose 0-cells are generated
by a single object under $\otimes$.


\section{Rewriting systems as 2-PROPS}
\label{sec:rewriting}

As remarked earlier we don't include any equations in our notion of
circuits 

%%%%%%%%%%%%%%%%%%%%%%%%%%%%%%%%%%%%%%%%%%%%%%%
%%        REWRITING SYSTEM DEFINITION        %%
%%%%%%%%%%%%%%%%%%%%%%%%%%%%%%%%%%%%%%%%%%%%%%%
\begin{definition}
  Given a free PROP of quantum circuits, we can extend it to a 2-PROP
    $\mathcal{P}$ where
  \begin{itemize}
    \item gates form the set of \emph{1-cells} 
    \item the \emph{2-cells} are generated by a finite set of \emph{rewriting}
      rules such that
    \begin{itemize}
      \item 2 rewriting rules with the same types must be equal
      \item each rewriting rule must be invertible, that is $\mathcal{C}(A, B)$
        is a \emph{groupoid} for all \emph{0-cells} $A, B$
    \end{itemize}
  \end{itemize}
  
  \emph{2-cells} are in general called \emph{rewrite sequence}, but if a rewrite
    sequence has only 1 occurence of a rewrite rule, it is called an
    \emph{atomic rule}.
\end{definition}

%%%%%%%%%%%%%%%%%%%%%%%%%%%%%%%%%%%%%%%%%%%%
%%        REWRITING SYSTEM EXAMPLE        %%
%%%%%%%%%%%%%%%%%%%%%%%%%%%%%%%%%%%%%%%%%%%%
\begin{example}
  Let's take the previous quantum circuit example \ref{ex:quantum-circs} and add
    some \emph{noisy gates}, that is, gates which can failed with an $\epsilon$
    probability.
  \[
    \begin{array}{cccccccccc}
      \circZ{{\epsilon}} & \qquad  & \circX{{\epsilon}} & \qquad &
        \circbox{H_\epsilon} \\ 
      \vstrut{1.2em}
      Z_\epsilon : 1 \to 1 && X_\epsilon : 1 \to 1 && H_\epsilon : 1 \to 1
    \end{array}
  \]
  We form a 2-PROP by adding some rewriting rules (we omit the identies)
  \[
    \begin{array}{cccccccc}
      \circZ{{\epsilon}} & \Rightarrow & \circZ{{}} & \alpha : Z_\epsilon \to Z \\
      \vstrut{1.2em} \\
      \circX{{\epsilon}} & \Rightarrow & \circX{{}} & \beta : X_\epsilon \to X \\
      \vstrut{1.2em} \\
      \circbox{H_\epsilon} & \Rightarrow & \circH & \gamma : H_\epsilon \to H
    \end{array}
  \]
  Then, on the circuit
  \[
    \begin{array}{cccccccccc}
      \circbox{H_\epsilon} \\ 
      \vstrut{1.2em} \\
      \circZ{{\epsilon}}
    \end{array}
  \]
  a rewriting sequence can be $(id_{H_\epsilon} \otimes \alpha) \bullet (\beta
    \otimes id_Z)$
  \[
    \begin{array}{cccccccccc}
      \circbox{H_\epsilon} & \overset{id_{H_\epsilon}}{\Rightarrow} &
        \circbox{H_\epsilon} & \overset{\beta}{\Rightarrow} & \circH \\ 
      \vstrut{1.2em} \\
      \circZ{{\epsilon}} & \overset{\alpha}{\Rightarrow} & \circZ{{}} &
        \overset{id_Z}{\Rightarrow} & \circZ{{}} \\ 
    \end{array}
  \]
  Moreover this sequence is atomic.
\end{example}

%%%%%%%%%%%%%%%%%%%%%%%%%%%%%%%%%%%%%%%%%%%%%%%%%%
%%        REWRITING SYSTEM PROPOSITION 1        %%
%%%%%%%%%%%%%%%%%%%%%%%%%%%%%%%%%%%%%%%%%%%%%%%%%%
\begin{proposition}
  For all rewriting sequences $\alpha: f \Rightarrow g$, $\alpha$ can be writen
  \[
    \alpha = \alpha_1 \bullet \alpha_2 \bullet \alpha_3 \bullet \cdots \bullet
      \alpha_i
  \]
  
  \begin{proof}
    We'll prove it by induction of the size of the rewriting sequence using the
      properties of the bifunctors.
    With a atomic rule, we already have
    \[
      \alpha = \alpha_1
    \]
    For the inductive step let's assume $\beta$ is
    \[
      \alpha = \alpha_1 \bullet \alpha_2 \bullet \alpha_3 \bullet \cdots \bullet
        \beta
    \]
    If $s$ is such that
    \[
      \beta = (\beta_1 \bullet \beta_2) \circ (\beta_3 \bullet \beta_4)
    \]
    Then using the interchange law we can write
    \[
      \beta = (\beta_1 \circ \beta_3) \bullet (\beta_2 \circ \beta_4)
    \]
    So we have 
    \begin{align}
      \alpha &= \alpha_1 \bullet \alpha_2 \bullet \alpha_3 \bullet \cdots
        \bullet (\beta_1 \circ \beta_3) \bullet (\beta_2 \circ \beta_4) \\
      \alpha &= \alpha_1 \bullet \alpha_2 \bullet \alpha_3 \bullet \cdots
        \bullet \alpha_{i - 1} \bullet \alpha_i
    \end{align}
    Otherwise we can always add an identity rewriting rule to get the
      interchange law.
    \begin{align}
      \beta_1 \circ (\beta_3 \bullet \beta_4) &\Rightarrow (\beta_1 \bullet id)
        \circ (\beta_2 \bullet \beta_3) \\
      (\beta_1 \bullet \beta_2) \circ \beta_3 &\Rightarrow (\beta_1 \bullet
        \beta_2) \circ (id \bullet \beta_3)
    \end{align}
    The same is true with
    \[
      \beta = (\beta_1 \bullet \beta_2) \otimes (\beta_3 \bullet \beta_4)
    \]
  \end{proof}
\end{proposition}

%%%%%%%%%%%%%%%%%%%%%%%%%%%%%%%%%%%%%%%%%%%%%%%%%%
%%        REWRITING SYSTEM PROPOSITION 2        %%
%%%%%%%%%%%%%%%%%%%%%%%%%%%%%%%%%%%%%%%%%%%%%%%%%%
\begin{proposition}
  For all rewriting sequences $\alpha: f \Rightarrow g$, $\alpha$ can be writen
  \begin{align}
    \alpha &= \alpha_1 \circ \alpha_2 \circ \alpha_3 \circ \cdots \circ \alpha_j
      \\
    \alpha &= \alpha_1 \otimes \alpha_2 \otimes \alpha_3 \otimes \cdots \otimes
      \alpha_k
  \end{align}
  
  \begin{proof}
    The proof is the same but this time we have to use
    \[
      (\alpha \otimes \beta) \circ (\gamma \otimes \delta) = (\alpha \circ
        \gamma) \otimes (\beta \circ \delta)
    \]
  \end{proof}
\end{proposition}

%%%%%%%%%%%%%%%%%%%%%%%%%%%%%%%%%%%%%%%%%%%%%%%%%%%%%%%%%%%%
%%        REWRITING SYSTEM PROPOSITION 2 COROLLARY        %%
%%%%%%%%%%%%%%%%%%%%%%%%%%%%%%%%%%%%%%%%%%%%%%%%%%%%%%%%%%%%
\begin{corollary}
  For all rewriting sequences $r: f \Rightarrow g$, $r$ can be writen
  \[
    \alpha = \alpha_1 \bullet \alpha_2 \bullet \alpha_3 \bullet \cdots \bullet
      \alpha_i
  \]
  Such that for all $i$
  \[
    \alpha_i = \alpha_{i, 1} \circ \alpha_{i, 2} \circ \alpha_{i, 3} \circ
      \cdots \circ \alpha_{i, j}
  \]
  and for all $j$
  \[
    \alpha_{i, j} = \bigotimes_k^l \alpha_{i, j, k}
  \]
  a tensor products of rewriting sequences.
\end{corollary}

\section{Quantale Enriched Categories}
\label{sec:quant-enrich-categ}


%%%%%%%%%%%%%%%%%%%%%%%%%%%%%%%%%%%%%%%
%%        QUANTALE DEFINITION        %%
%%%%%%%%%%%%%%%%%%%%%%%%%%%%%%%%%%%%%%%
\begin{definition}[Quantale $\mathbb{Q}$]
  A quantale is a \emph{lattice} with a binary operator such that for all
    elements of the lattice $x, y$ and $\{x_i\}_i, \{y_i\}_i$
  \begin{align}
    x \oplus \bigvee y_i &= \bigvee x \oplus y_i \\
    \left(\bigvee x_i\right) \oplus y &= \bigvee x_i \oplus y
  \end{align}
  
  We require that for all $x$, it exist $x^{-1}$ such that
  \begin{align}
    x \oplus x^{-1} &= e \\
    x^{-1} \oplus x &= e
  \end{align}
  
  Such a quantale can be seen as a monoidal category over a \emph{group}.
\end{definition}

%%%%%%%%%%%%%%%%%%%%%%%%%%%%%%%%%%%%%%%%%%%%%%%%%%
%%        Q-ENRICHED CATEGORY DEFINITION        %%
%%%%%%%%%%%%%%%%%%%%%%%%%%%%%%%%%%%%%%%%%%%%%%%%%%
\begin{definition}[$\mathbb{Q}$-enriched category]
  A $\mathbb{Q}$-enriched category $\mathcal{E}$ is
  
  \begin{itemize}
    \item A set of objects
    \item For all objects $(a, b)$, $\mathcal{E}(a, b) \in \mathbb{Q}$
  \end{itemize}
  
  such that for all objects $a$
  \[
    e \succeq \mathcal{E}(a, a)
  \]
  for all objects (a, b, c)
  \[
    \mathcal{E}(b, c) \oplus \mathcal{E}(a, b) \succeq \mathcal{E}(a, c)
  \]
\end{definition}

%%%%%%%%%%%%%%%%%%%%%%%%%%%%%%%%%%%%%%%%%%%%%%%%%%%%%%%%%%%
%%        MONOIDAL Q-ENRICHED CATEGORY DEFINITION        %%
%%%%%%%%%%%%%%%%%%%%%%%%%%%%%%%%%%%%%%%%%%%%%%%%%%%%%%%%%%%
\begin{definition}[Monoidal $\mathbb{Q}$-enriched category]
  $(E, \otimes)$ is strictly monoidal if
  \[
    \mathcal{E}(a, b) \oplus \mathcal{E}(a', b') \succeq \mathcal{E}(a \otimes
      a', b \otimes b')
  \]
  and symmetric if
  \[
    e \succeq \mathcal{E}(a \otimes b, b \otimes a)
  \]
\end{definition}



\section{$\mathbb{Q}$-valued rewriting systems}

%%%%%%%%%%%%%%%%%%%%%%%%%%%%%%%%%%%%%%%%%%%
%%        HOMOMORPHISM DEFINITION        %%
%%%%%%%%%%%%%%%%%%%%%%%%%%%%%%%%%%%%%%%%%%%
\begin{definition}
  For all $(A, B) \in \mathcal{P}$ we define a homomorphism $V:
    \mathcal{C}(A, B) \to \mathbb{Q}$, such that for all gates $f$
  \[
    V(f) \in \mathbb{Q}
  \]
  For all rewriting rules $\alpha: f \Rightarrow g$
  \[
    V(\alpha) = V(f) \oplus V(g)^{-1}
  \]
  
  On gates
  \begin{align}
    V(f \circ g) &\preceq V(f) \oplus V(g) \\
    V(f \otimes g) &\preceq V(f) \oplus V(g)
  \end{align}
  
  On rewriting sequences
  \begin{align}
    V(\alpha \circ \beta) &= V(\alpha) \oplus V(\beta) \\
    V(\alpha \otimes \beta) &= V(\alpha) \oplus V(\beta) \\
    V(\alpha \bullet \beta) &= V(\alpha) \oplus V(\beta)
  \end{align}
\end{definition}

%%%%%%%%%%%%%%%%%%%%%%%%%%%%%%%%%%%%%%%
%%        HOMOMORPHISM REMARK        %%
%%%%%%%%%%%%%%%%%%%%%%%%%%%%%%%%%%%%%%%
\begin{remark}
  Note that because
  \[
    V(\alpha: f \Rightarrow g) = V(f) \oplus V(g)^{-1}
  \]
  then
  \begin{align}
    V(\alpha: f \Rightarrow f) &= V(f) \oplus V(f)^{-1} \\
    &= e
  \end{align}
  that is
  \[
    \forall f, V(id_f) = e
  \]
\end{remark}

%%%%%%%%%%%%%%%%%%%%%%%%%%%%%%%%%%%%%%%%%%%%%%
%%        HOMOMORPHISM PROPOSITION 1        %%
%%%%%%%%%%%%%%%%%%%%%%%%%%%%%%%%%%%%%%%%%%%%%%
\begin{proposition}
  Given two rewriting sequence $\alpha, \beta: f \Rightarrow g$, then $V(\alpha)
    = V(\beta)$.
  
  \begin{proof}
    The set of all possible rewriting sequences is freely generated by the set
      of rewriting rules, $\bullet$, $\circ$ and $\otimes$ so both $\alpha$ and
      $\beta$ can be written
    \begin{align}
      \alpha &= \alpha_1 \odot \alpha_2 \odot \alpha_3 \odot \cdots \odot
        \alpha_n \\
      \beta &= \beta_1 \odot \beta_2 \odot \beta_3 \odot \cdots \odot \beta_m
    \end{align}
    with $\odot \in \{\bullet, \circ, \otimes\}$ and for all i $\alpha_i$ and
      $\beta_i$ are rewriting rule.
    If $n = 1$ then $\alpha$ is a rewriting rule, so $V(r) = V(f) \oplus
      V(g)^{-1}$.
    Otherwise
    \begin{align}
      V(\alpha) &= V(\alpha_1 \odot \alpha_2 \odot \cdots \odot \alpha_n) \\
      &= V(\alpha_1) \oplus V(\alpha_2) \oplus \cdots \oplus V(\alpha_n) \\
      &= V(f \Rightarrow h_1) \oplus V(h_1 \Rightarrow h_2) \oplus \cdots \oplus
        V(h_n \Rightarrow g) \\
      &= V(f) \oplus V(h_1)^{-1} \oplus V(h_1) \oplus V(h_2)^{-1} \oplus \cdots
        \oplus V(h_n) \oplus V(g)^{-1}
    \end{align}
    A easy inductive proof can show us that
    \[
      V(f) \oplus V(h_1)^{-1} \oplus V(h_1) \oplus V(h_2)^{-1} \oplus V(h_2)
        \oplus \cdots \oplus V(h_n) \oplus V(g)^{-1} = V(f) \oplus V(g)^{-1}
    \]
    The same is true for $\beta$, so
    \[
      V(\alpha) = V(\beta) = V(f) \oplus V(g)^{-1}
    \]
  \end{proof}
\end{proposition}

%%%%%%%%%%%%%%%%%%%%%%%%%%%%%%%%%%%%%%%%%%%%%%%%%%%%%%%%
%%        HOMOMORPHISM PROPOSITION 1 COROLLARY        %%
%%%%%%%%%%%%%%%%%%%%%%%%%%%%%%%%%%%%%%%%%%%%%%%%%%%%%%%%
\begin{corollary}
  For all rewriting sequences $\alpha: f \Rightarrow g$, then $V(\alpha) = V(f)
    \oplus V(g)^{-1}$.
  Note that the definition of $V$ gives that only for rewriting rules.
\end{corollary}

%%%%%%%%%%%%%%%%%%%%%%%%%%%%%%%%%%%%%%%%%%%%%%
%%        HOMOMORPHISM PROPOSITION 2        %%
%%%%%%%%%%%%%%%%%%%%%%%%%%%%%%%%%%%%%%%%%%%%%%
\begin{proposition}
  The following category $\mathcal{P}(\mathbb{Q})$ is a $\mathbb{Q}$-category
  \begin{itemize}
    \item A set of objects which are the gates of $\mathcal{P}$
    \item For all objects $(f, g)$
  \end{itemize}
  \[
    \mathcal{E}(f, g) =
    \begin{cases}
      \text{if } f \Rightarrow g \in \mathcal{P}, &V(f \Rightarrow g) \\
      \text{otherwise } &\top
    \end{cases}
  \]
  With $\top \in \mathbb{Q}$ such that for all $x \in \mathbb{Q}$
  \[
    x \preceq \top
  \]
  
  \begin{proof}
    For all gates $(f, g)$, if it exists a unique transformation $f \Rightarrow
      g$, then
    \begin{align}
      \mathcal{E}(f, g) &= V(f \Rightarrow g) \in \mathbb{Q} \\
      &= V(f) \oplus V(g)^{-1}
    \end{align}
    otherwise we already have $\mathcal{E}(f, g) = \top \in \mathbb{Q}$.
    For all gates $f$, we know that there is a identity transformation $f
      \Rightarrow f$ and
    \begin{align}
      \mathcal{E}(f, f) &= V(id_f) \\
      &= V(f \Rightarrow f)\\
      &= V(f) \oplus V(f)^{-1} \\
      &= e
    \end{align}
    Finally, for all gates $(f, g, h)$ such that there exists $\alpha: f
      \Rightarrow g$ and $\beta: g \Rightarrow h$ in $\mathcal{P}$, then the
      homomorphism $V$ give us
    \begin{align}
      \mathcal{E}(f, h) &= V(f \Rightarrow h) \\
      &= V(\alpha \circ \beta) \\
      &= V(\alpha) \oplus V(\beta) \\
      &= V(g \Rightarrow h) \oplus V(f \Rightarrow g) \\
      &= \mathcal{E}(g, h) \oplus \mathcal{E}(f, g)
    \end{align}
    If either $\alpha$ or $\beta$ doesn't exists in $\mathcal{P}$ then
    \[
      \mathcal{E}(b, c) \oplus \mathcal{E}(a, b) \succeq \mathcal{E}(a, c)
    \]
    become
    \[
      \begin{matrix}
        &\mathcal{E}(b, c) \oplus \top &\succeq \top \\
        \text{or} &\top \oplus \mathcal{E}(a, b) &\succeq \top \\
        \text{or} &\top \oplus \top &\succeq \top
      \end{matrix}
    \]
    which is still true.
  \end{proof}
\end{proposition}

%%%%%%%%%%%%%%%%%%%%%%%%%%%%%%%%%%%%%%%%%%%%%%
%%        HOMOMORPHISM PROPOSITION 3        %%
%%%%%%%%%%%%%%%%%%%%%%%%%%%%%%%%%%%%%%%%%%%%%%
\begin{proposition}
  $\mathcal{P}(\mathbb{Q})$ is strictly symmetric monoidal with $\otimes$, the
    2-functor of $\mathcal{P}$.
  
  \begin{proof}
    The monoidal structure is immediate using the fact that
    \[
      V(\alpha) \oplus V(\beta) \succeq V(\alpha \otimes \beta)
    \]
    that is
    \[
      V(\alpha: f \Rightarrow g) \oplus V(\beta: f' \Rightarrow g') = V(\alpha
        \otimes \beta: f \otimes f' \Rightarrow g \otimes g')
    \]
    so
    \[
      \mathcal{E}(f, g) \oplus \mathcal{E}(f', g') = \mathcal{E}(f \otimes f', g
        \otimes g')
    \]
    For the symmetry, we use the fact that $V(f \Rightarrow g) \oplus V(g
      \Rightarrow f) = e$
    \begin{align}
      \mathcal{E}(f \otimes g, g \otimes f) &= V(f \otimes g \Rightarrow g
        \otimes g) \\
      &= V(f \Rightarrow g \otimes g \Rightarrow f) \\
      &\preceq V(f \Rightarrow g) \oplus V(g \Rightarrow f) \\
      &= V(f) \oplus V(g)^{-1} \oplus V(g) \oplus V(f)^{-1} \\
      &= e
    \end{align}
  \end{proof}
\end{proposition}

%%%%%%%%%%%%%%%%%%%%%%%%%%%%%%%%%%%%%%%%%%%%%%%%%%%%%%%%
%%        HOMOMORPHISM PROPOSITION 3 COROLLARY        %%
%%%%%%%%%%%%%%%%%%%%%%%%%%%%%%%%%%%%%%%%%%%%%%%%%%%%%%%%
\begin{corollary}
  Given $\alpha: f \Rightarrow g$ a rewriting sequence, and $\alpha^{-1}: g
    \Rightarrow f$ the inverse rewriting sequence
  \[
    V(s \otimes s^{-1}) = e
  \]
\end{corollary}


\section{Conclusions and further work}
\label{sec:conclusion}



\small
\bibliography{all}


\end{document}
