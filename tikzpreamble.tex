\usepackage{tikz}
\usepackage{pgfplots}
\usetikzlibrary{trees}
\usetikzlibrary{topaths}
\usetikzlibrary{decorations.pathmorphing}
\usetikzlibrary{decorations.markings}
\usetikzlibrary{matrix,backgrounds,folding}
\usetikzlibrary{chains,scopes,positioning,fit}
\usetikzlibrary{arrows,shadows}
\usetikzlibrary{calc} 
\usetikzlibrary{chains}
\usetikzlibrary{shapes,shapes.geometric,shapes.misc}

\usepackage[undirected]{tikzquanto}
\usepackage{tikzcircuits}

% \newcommand{\tikzfig}[1]{\inline{%
% \beginpgfgraphicnamed{#1}
% \InputIfFileExists{#1.tikz}{}{\input{./tikz/#1.tikz}}
% \endpgfgraphicnamed}}

% \newcommand{\ctikzfig}[1]{%
% \begin{center}\rm
%   \tikzfig{#1}
% \end{center}}

%\newcommand{\inlinetikzfig}[1]{\InputIfFileExists{#1.tikz}{}{\input{./tikz/#1.tikz}}}

\pgfdeclarelayer{edgelayer}
\pgfdeclarelayer{nodelayer}
\pgfsetlayers{background,edgelayer,nodelayer,main}

% this is the "quanto" style
\tikzstyle{every picture}=[baseline=(current bounding box).east,node distance=5mm]
\tikzstyle{none}=[inner sep=0mm]
\tikzstyle{every loop}=[]

% this is a work around for tikzit bug
\tikzstyle{(null)}=[]
\tikzstyle{plain}=[]

%\tikzstyle{every loop}=[]

% \tikzstyle{cloud} = [draw=gray!18,fill=gray!10, ellipse, node
% distance=3cm,minimum height=3em, minimum width=5em]

% some predefined tikz drawings
%\input{tikz/gens.tex}



%%% This by Liam -- think it's not needed.
\tikzset{
operator/.style = {draw,fill=white,minimum size=1.5em},%
operator2/.style ={draw,fill=white,minimum height=7cm,minimum width=1.5 cm},%
phase/.style = {draw,fill,shape=circle,minimum size=5pt,inner sep=0pt},% 
endline/.style = {draw,fill,shape=circle,minimum size=1pt,inner sep=0pt},%
surround/.style = {fill=blue!10,thick,draw=black,rounded corners=2mm},%
cross/.style={},
    % path picture={ \draw[thick,black](path picture bounding%
    % box.north) -- (path picture bounding box.south) (path picture%
    % bounding box.west) -- (path picture bounding box.east);}},%
circlewc/.style={draw,circle,cross,minimum width=0.3 cm},%
} %% end tikzset
